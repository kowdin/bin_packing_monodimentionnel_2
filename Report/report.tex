\documentclass{article}

\usepackage[utf8]{inputenc}
\usepackage[T1]{fontenc}
\usepackage[frenchb]{babel}

\usepackage{a4wide}

\usepackage{amsmath}
\usepackage{amssymb}
\usepackage{amsthm}

\usepackage{hyperref}


\title{Continuous optimization}
\author{Samuel Buchet \& Dorian Dumez \& Brendan Guevel}
\date{Mai 2017}

\begin{document}

\maketitle

\section{Relaxation continue}

\section{tâches}

\begin{itemize}
    \item réaliser un parseur
    \item faire la relaxation continue
    \item heuristique pour borne supérieure (best fit)
    \item relaxation lagrangienne + réparation
    \item tests statistiques
\end{itemize}
~\\
Pour l'heuristique lagrangienne :
\begin{itemize}
    \item choisir les contraintes à dualiser
    \item trouver un moyen de résoudre le problème relâché
    \item mettre en oeuvre l'algorithme de descente de gradient
    \item créer une heuristique de réparation
\end{itemize}

\end{document}
